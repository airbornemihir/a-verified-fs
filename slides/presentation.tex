% titlepage-demo.tex
\documentclass{beamer}

% items enclosed in square brackets are optional; explanation below
\title{Verifying filesystems in ACL2}
\subtitle{Towards verifying file recovery tools}
\author{Mihir Mehta}
\institute{
  Department of Computer Science\\
  University of Texas at Austin\\[1ex]
  \texttt{mihir@cs.utexas.edu}
}
\date{09 March, 2017}

\begin{document}

%--- the titlepage frame -------------------------%
\begin{frame}[plain]
  \titlepage
\end{frame}

%--- the presentation begins here ----------------%
\begin{frame}{Overview}
  \begin{enumerate}
  \item Why we need a verified filesystem
  \item Our approach
  \item Progress so far
  \item Future work
  \end{enumerate}
\end{frame}

\begin{frame}{Why we need a verified filesystem}
  \begin{itemize}
  \item Filesystems are everywhere.
  \item Yet they're poorly understood - especially by people who
    should.
  \item Modern filesystems have become increasingly complex, and so
    have the tools to analyse and recover data from them.
  \item It might be nice, it might be nice to verify that the
    filesystems and the tools actually provide the guarantees they
    claim to provide.
  \end{itemize}
\end{frame}

\begin{frame}{Our approach}
  \begin{itemize}
  \item Build a series of models, each providing a minimal set of
    operations and proofs of correctness of these operations.
  \item Per Ghemawat et al in SOSP 2003, a minimal set of operations
    can suffice - create, delete, open, close, read, and write files.
  \item Increasing the complexity of these operations with each model
    while proving equivalence with the previous model as we go would
    make the proofs tractable.
  \item There are many properties that could be considered for
    correctness, but the read-over-write theorems from the first-order
    theory of arrays seem like a good place to start.
    \begin {enumerate}
    \item Reading from a location after writing to the same location
      should yield the data that was written.
    \item Reading from a location after writing to a different
      location should yield the same result as reading before writing.
    \end {enumerate}
  \end{itemize}
\end{frame}

\begin{frame}{Progress so far}
  \begin{itemize}
    \item We've built three models, with the operations read, write,
      and delete.
      \begin{enumerate}
      \item In the first model, we represent the filesystem as a tree,
        allow for text files (stored as strings) and directories only,
        and store no metadata.
      \item In the second model, we add some metadata to our tree
        representation - namely, file length. We introduce a
        rudimentary fsck and prove that our operations of writing and
        deleting preserve correctness under fsck.
      \item In the third model, we retain the tree but also introduce
        an unlimited "disk" of fixed-length character blocks. We do
        away with the explicit strings holding the contents of text
        files and instead store lists of block indices in the tree.
      \end{enumerate}
    \item For each of these models, we have proofs of correctness of
      the two read-after-write properties, based on the proofs of
      equivalence between each model and its successor.
  \end{itemize}
\end{frame}

\begin{frame}{Future work}
  \begin{itemize}
  \item Add the create operation, and possibly open and close with the
    introduction of file descriptors in future models.
  \item In the next model, finitise the length of the disk and garbage
    collect disk blocks that are left unused after a write or a delete
    operation.
  \item In future models, linearise the tree, leaving only the disk.
  \item Eventually emulate the CP/M filesystem as a convincing proof
    of concept, and move on to fsck and file recovery tools.
  \end{itemize}
\end{frame}

\end{document}

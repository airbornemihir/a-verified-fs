\documentclass[format=sigconf,review=true]{acmart}
\title{Verifying file systems with ACL2}
\subtitle{Towards verifying data recovery tools}
\author{Mihir P. Mehta}

\affiliation{%
  \institution{University of Texas at Austin}
  \city{Austin}
  \state{TX}
  \country{USA}}
\email{mihir@cs.utexas.edu}

\thanks{This work is supported by a grant from the NSF. }

\begin{document}

\maketitle

\section{Introduction}

In this paper, we describe work in progress to model and verify
filesystems using the ACL2 theorem prover.

\section{Motivation}
Filesystems are ubiquitous, and a critical factor in the security and
performance of all applications. Yet, they remain poorly understood,
a problem which has been exacerbated by the complexity of modern
filesystems which use redundancy and caching in order to be faster and
more reliable. As a consequence, many tools which interact deeply with
the filesystem, such as file deletion and file recovery tools, have
become more vulnerable to bugs because of the complexity of these
tasks. Thus, it is worthwhile to work towards formally verifying the
guarantees provided by a filesystem.

\section{Modelling a filesystem}

In order to make our proofs of correctness tractable, we choose to
make several verified filesystem models in increasing order of
complexity. This approach supports incremental proof strategies,
providing us with a choice between proving a model equivalent to the
next, and simply adapting existing proofs for the next model.

While starting out, we faced a decision about the file system
operations we should provide. We decided against implementing the
entirety of the Linux VFS interface, reasoning that this would require
us to implement 19 inode operations, 6 dentry operations and 22 file
operations. Following the  example of the Google File System
(citation), we decided to restrict ourselves to a small
number of fundamental file system operations - namely reading, writing,
creating, and deleting a file. This excludes the operations of opening
and closing a file; we hope to implement these when they become
necessary for verification in a multiprogramming environment.

\section{Model 1}
The intuitive mental model of a filesystem is a tree, which remains
useful even though it fails for filesystems with links. Accordingly,
it is appropriate for our first model, which will serve as a
specification for all later models, to be a literal tree. Our
filesystem recogniser, \texttt{l1-fs-p}, recognises
\texttt{symbol-alist}s where each cdr of a pair in the alist satisfies
either \texttt{stringp} (denoting a regular file) or \texttt{l1-fs-p}
(denoting a subdirectory).

\section{Model 2}
Model 1 can hold unbounded text files and nested directory
structures. However, real filesystems include metadata, and including
metadata in our filesystem representation also allows us to define a
notion of "consistency" wherein the actual contents of a regular or
directory file are checked for agreement with the metadata. Thus, in
our next model, we add an extra field for length of a
regular file. We also create a simple version of fsck that checks
file contents for consistency with the stated length, and verify
that the operations for writing, creating and deleting preserve this
notion of consistency.

\section{Model 3}
Next, we would like to move towards a more realistic file storage
paradigm where the contents of a regular file are broken into
fixed-size blocks and stored in an external table, which we will refer
to as the disk. In this model, we store the text of a regular file in
the disk, and retain only the indices of the relevant blocks in the
filesystem tree. For now, we consider the disk to be unbounded and
make no attempt at garbage collection. Thus, file creation and writing
operations can be represented as append operations, where the new
blocks representing the new contents of a file are simply placed at
the end of the disk with no effort to free the old blocks or erase
their contents. Similarly, deleting a file does not require any disk
operations; the blocks of such a file remain in the disk but are no longer
referred to.

\section{Model 4}
In this model, we finitise our disk; this necessitates garbage
collection which we approximate through reference
counting. Since we allow neither symbolic links nor hard links in our
filesystem, the reference count of any block in the disk is either 0
or 1; this allows us to implement reference counting through an
allocation vector, i.e. an array of booleans with the same length as
the disk. Thus, in every write or delete operation, the allocation
vector entries corresponding to blocks which are no longer used must
be marked free; similarly, in every write or create operation, the
allocation vector must be scanned to find the appropriate number of
free blocks. The lockstep updates described here allow us to prove
that aliasing between different files does not occur.

\section {Proof approach}

Initially, we would like to prove two well-known properties from the
first-order theory of arrays, namely the read-over-write properties.

\begin {enumerate}
\item Reading from a location after writing to the same location
  should yield the data that was written.
\item Reading from a location after writing to a different
  location should yield the same result as reading before writing.
\end {enumerate}

\section{Future work}
As previously mentioned, we would like to add the system calls open and close with the
introduction of file descriptors. This would be a step towards the
study of concurrent FS operations. We would also like to linearise
the tree, leaving only the disk - this would be more in keeping
with realistic file systems that do not require an in-memory tree
representation, but still allow tree traversal through systematic
lookups in the disk.

Eventually, we would like to emulate the CP/M filesystem as a convincing proof
of concept. This would be a step towards verified versions of fsck
and file recovery tools, which could be based on our proofs about
the underlying filesystem.

\section{Related work}
Currently, the state of the art is represented by Haogang
Chen's dissertation work (reference), in which the author uses Coq to
build a filesystem (named FSCQ) which is proven safe against
crashes. This implementation was exported into Haskell, and showed
comparable performance to ext4 when run on the Linux kernel
through the FUSE layer.

Our work takes a different approach - our aim is to produce verified
models of existing filesystems that have binary compatibility with the
filesystem layout read and written by the corresponding
implementation. This allows us to find bugs in existing filesystems,
which is not addressed by Chen's work.

\end{document}

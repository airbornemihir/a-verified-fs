\documentclass[submission,copyright,creativecommons]{eptcs}
\providecommand{\event}{SOS 2007} % Name of the event you are submitting to
\usepackage{breakurl}             % Not needed if you use pdflatex only.
\usepackage{underscore}           % Only needed if you use pdflatex.

\title{An Example of a Paper\\ with a Rather Large Title-to-Content Ratio}
\author{Rob van Glabbeek
\institute{NICTA\\ Sydney, Australia}
\institute{School of Computer Science and Engineering\\
University of New South Wales\thanks{A fine university.}\\
Sydney, Australia}
\email{rvg@cs.stanford.edu}
\and
Co Author \qquad\qquad Yet S. Else
\institute{Stanford Univeristy\\
California, USA}
\email{\quad is@gmail.com \quad\qquad somebody@else.org}
}
\def\titlerunning{A Longtitled Paper}
\def\authorrunning{R.J. van Glabbeek, C. Author \& Y.S. Else}
\begin{document}
\maketitle

\begin{abstract}
This is a sentence in the abstract.
This is another sentence in the abstract.
This is yet another sentence in the abstract.
This is the final sentence in the abstract.
\end{abstract}

\section{Introduction}

Filesystems provide a convenient interface for the named storage of
data, and as such are ubiquitous in application programs. Filesystem
bugs expose users to the risks of data breach, corruption, loss; thus,
verifying their correctness is essential. In this work, we present an
approach towards constructing executable specifications of existing
filesystems and verifying their functional properties in a theorem
proving environment. We detail an application of this approach to the
FAT32 filesystem.

We also detail the methodology used to build up this type of
executable specification through a series of models which
incrementally add features of the target filesystem. This methodology
has the benefit of allowing the verification effort to start from
simple models which encapsulate features common to many filesystems
and are thus suitable for re-use.

\section{Ancillary files}

Authors may upload ancillary files to be linked alongside their paper.
These can for instance contain raw data for tables and plots in the
article, or program code.  Ancillary files are included with an EPTCS
submission by placing them in a directory \texttt{anc} next to the
main latex file. See also \url{https://arxiv.org/help/ancillary_files}.
Please add a file README in the directory \texttt{anc}, explaining the
nature of the ancillary files, as in
\url{http://eptcs.org/paper.cgi?226.21}.

\section{Prefaces}

Volume editors may create prefaces using this very template,
with {\tt $\backslash$title$\{$Preface$\}$} and {\tt $\backslash$author$\{\}$}.

\section{Bibliography}

We request that you use
\href{http://eptcs.web.cse.unsw.edu.au/eptcs.bst}
{\tt $\backslash$bibliographystyle$\{$eptcs$\}$}
\cite{bibliographystylewebpage}, or one of its variants
\href{http://eptcs.web.cse.unsw.edu.au/eptcsalpha.bst}{eptcsalpha},
\href{http://eptcs.web.cse.unsw.edu.au/eptcsini.bst}{eptcsini} or
\href{http://eptcs.web.cse.unsw.edu.au/eptcsalphaini.bst}{eptcsalphaini}
\cite{bibliographystylewebpage}. Compared to the original {\LaTeX}
{\tt $\backslash$biblio\-graphystyle$\{$plain$\}$},
it ignores the field {\tt month}, and uses the extra
bibtex fields {\tt eid}, {\tt doi}, {\tt ee} and {\tt url}.
The first is for electronic identifiers (typically the number $n$
indicating the $n^{\rm th}$ paper in an issue) of papers in electronic
journals that do not use page numbers. The other three are to refer,
with life links, to electronic incarnations of the paper.

\paragraph{DOIs}

Almost all publishers use digital object identifiers (DOIs) as a
persistent way to locate electronic publications. Prefixing the DOI of
any paper with {\tt http://dx.doi.org/} yields a URI that resolves to the
current location (URL) of the response page\footnote{Nowadays, papers
  that are published electronically tend
  to have a \emph{response page} that lists the title, authors and
  abstract of the paper, and links to the actual manifestations of
  the paper (e.g.\ as {\tt dvi}- or {\tt pdf}-file). Sometimes
  publishers charge money to access the paper itself, but the response
  page is always freely available.}
of that paper. When the location of the response page changes (for
instance through a merge of publishers), the DOI of the paper remains
the same and (through an update by the publisher) the corresponding
URI will then resolve to the new location. For that reason a reference
ought to contain the DOI of a paper, with a life link to the corresponding
URI, rather than a direct reference or link to the current URL of
publisher's response page. This is the r\^ole of the bibtex field {\tt doi}.
{\bf EPTCS requires the inclusion of a DOI in each cited paper, when available.}

DOIs of papers can often be found through
\url{http://www.crossref.org/guestquery};\footnote{For papers that will appear
  in EPTCS and use \href{http://eptcs.web.cse.unsw.edu.au/eptcs.bst}
  {\tt $\backslash$bibliographystyle$\{$eptcs$\}$} there is no need to
  find DOIs on this website, as EPTCS will look them up for you
  automatically upon submission of a first version of your paper;
  these DOIs can then be incorporated in the final version, together
  with the remaining DOIs that need to found at DBLP or publisher's webpages.}
the second method {\it Search on article title}, only using the {\bf
surname} of the first-listed author, works best.  
Other places to find DOIs are DBLP and the response pages for cited
papers (maintained by their publishers).

\paragraph{The bibtex fields {\tt ee} and {\tt url}}

Often an official publication is only available against payment, but
as a courtesy to readers that do not wish to pay, the authors also
make the paper available free of charge at a repository such as
\url{arXiv.org}. In such a case it is recommended to also refer and
link to the URL of the response page of the paper in such a
repository.  This can be done using the bibtex fields {\tt ee} or {\tt
url}, which are treated as synonyms.  These fields should \textbf{not} be used
to duplicate information that is already provided through the DOI of
the paper.
You can find archival-quality URL's for most recently published papers
in DBLP---they are in the bibtex-field {\tt ee}---but please suppress
repetition of DOI information. In fact, it is often
useful to check your references against DBLP records anyway, or just find
them there in the first place.

\paragraph{Typesetting DOIs and URLs}

When using {\LaTeX} rather than {\tt pdflatex} to typeset your paper, by
default no linebreaking within long URLs is allowed. This leads often
to very ugly output, that moreover is different from the output
generated when using {\tt pdflatex}. This problem is repaired when
invoking \href{http://eptcs.web.cse.unsw.edu.au/breakurl.sty}
{\tt $\backslash$usepackage$\{$breakurl$\}$}: it allows linebreaking
within links and yield the same output as obtained by default with
{\tt pdflatex}. 
When invoking {\tt pdflatex}, the package {\tt breakurl} is ignored.

Please avoid using {\tt $\backslash$usepackage$\{$doi$\}$}, or
{\tt $\backslash$newcommand$\{\backslash$doi$\}$}.
If you really need to redefine the command {\tt doi}
use {\tt $\backslash$providecommand$\{\backslash$doi$\}$}.

The package {\tt $\backslash$usepackage$\{$underscore$\}$} is
recommended to deal with underscores in DOIs. This is not needed when
using {\tt $\backslash$usepackage$\{$breakurl$\}$} and not {\tt pdflatex}.

\paragraph{References to papers in the same EPTCS volume}

To refer to another paper in the same volume as your own contribution,
use a bibtex entry with
\begin{center}
  {\tt series    = $\{\backslash$thisvolume$\{5\}\}$},
\end{center}
where 5 is the submission number of the paper you want to cite.
You may need to contact the author, volume editors or EPTCS staff to
find that submission number; it becomes known (and unchangeable)
as soon as the cited paper is first uploaded at EPTCS\@.
Furthermore, omit the fields {\tt publisher} and {\tt volume}.
Then in your main paper you put something like:

\noindent
{\small \tt $\backslash$providecommand$\{\backslash$thisvolume$\}$[1]$\{$this
  volume of EPTCS, Open Publishing Association$\}$}

\noindent
This acts as a placeholder macro-expansion until EPTCS software adds
something like

\noindent
{\small \tt $\backslash$newcommand$\{\backslash$thisvolume$\}$[1]%
  $\{\{\backslash$eptcs$\}$ 157$\backslash$opa, pp 45--56, doi:\dots$\}$},

\noindent
where the relevant numbers are pulled out of the database at publication time.
Here the newcommand wins from the providecommand, and {\tt \small $\backslash$eptcs}
resp.\ {\tt \small $\backslash$opa} expand to

\noindent
{\small \tt $\backslash$sl Electronic Proceedings in Theoretical Computer Science} \hfill and\\
{\small \tt , Open Publishing Association} \hfill .

\noindent
Hence putting {\small \tt $\backslash$def$\backslash$opa$\{\}$} in
your paper suppresses the addition of a publisher upon expansion of the citation by EPTCS\@.
An optional argument like
\begin{center}
  {\tt series    = $\{\backslash$thisvolume$\{5\}[$EPTCS$]\}$},
\end{center}
overwrites the value of {\tt \small $\backslash$eptcs}.

\nocite{*}
\bibliographystyle{eptcs}
\bibliography{references}
\end{document}

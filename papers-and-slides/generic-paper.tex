
%%%%%%%%%%%%%%%%%%%%%%% file typeinst.tex %%%%%%%%%%%%%%%%%%%%%%%%%
%
% This is the LaTeX source for the instructions to authors using
% the LaTeX document class 'llncs.cls' for contributions to
% the Lecture Notes in Computer Sciences series.
% http://www.springer.com/lncs       Springer Heidelberg 2006/05/04
%
% It may be used as a template for your own input - copy it
% to a new file with a new name and use it as the basis
% for your article.
%
% NB: the document class 'llncs' has its own and detailed documentation, see
% ftp://ftp.springer.de/data/pubftp/pub/tex/latex/llncs/latex2e/llncsdoc.pdf
%
%%%%%%%%%%%%%%%%%%%%%%%%%%%%%%%%%%%%%%%%%%%%%%%%%%%%%%%%%%%%%%%%%%%


\documentclass[runningheads,a4paper]{llncs}

\usepackage{amssymb}
\setcounter{tocdepth}{3}
\usepackage{graphicx}
\usepackage{graphicx}
\usepackage{tikz}
\usetikzlibrary{matrix}

\usepackage{url}
\urldef{\mailsa}\path|mihir@cs.utexas.edu|
%% \urldef{\mailsa}\path|{mihir,|
%% \urldef{\mailsc}\path|hunt}@cs.utexas.edu|
\newcommand{\keywords}[1]{\par\addvspace\baselineskip
\noindent\keywordname\enspace\ignorespaces#1}

\begin{document}

\mainmatter  % start of an individual contribution

% first the title is needed
\title{Verifying fileutils in ACL2: a case study}

% a short form should be given in case it is too long for the running head
\titlerunning{fileutils verification}

% the name(s) of the author(s) follow(s) next
%
% NB: Chinese authors should write their first names(s) in front of
% their surnames. This ensures that the names appear correctly in
% the running heads and the author index.
%
\author{Mihir Parang Mehta}
%
\authorrunning{M. P. Mehta}
% (feature abused for this document to repeat the title also on left hand pages)

% the affiliations are given next; don't give your e-mail address
% unless you accept that it will be published
\institute{University of Texas at Austin, Department of Computer Science,\\
2317 Speedway, Austin, TX 78712, USA\\}

%
% NB: a more complex sample for affiliations and the mapping to the
% corresponding authors can be found in the file "llncs.dem"
% (search for the string "\mainmatter" where a contribution starts).
% "llncs.dem" accompanies the document class "llncs.cls".
%

\toctitle{ACL2 filesystem verification}
\tocauthor{Mihir Mehta}
\maketitle

\begin{abstract}
  We describe an effort to verify the \texttt{fileutils} subset of the
  GNU \texttt{coreutils} against specifications built upon a verified
  model of the FAT32 filesystem.
\keywords{interactive theorem proving, filesystems}
\end{abstract}

\section{Introduction and overview}

The \texttt{fileutils} are neat. They would be neater if formally verified.

\section{Related work}

Filesystem verification research has largely followed a pattern of
synthesising a new filesystem based on a specification chosen for its
ease in proving properties of interest, rather than similarity to an
existing filesystem. FSCQ \cite{DBLP:conf/usenix/ChenZCCKZ16} is an
example.

\section{Evaluation}

We specify and verify all the utilities in the \texttt{fileutils}
subset of \texttt{coreutils}.

\section{Conclusion}

This work shows that a formal model of a single filesystem can be used
to verify application programs with non-trivial interactions with the
filesystem. Additionally, this work provides library support for
working with additional filesystems and possibly identifying
differences between different filesystems when used with the same
program.

\section{Future work}

We hope to expand on this work by specifying and verifying the
operation of application programs in multiprogramming environments
where concurrent accesses to the filesystem may be made by different
processes.

\subsubsection*{Acknowledgments.} This material is based upon work
supported by the National Science Foundation SaTC program under
contract number CNS-1525472. Thanks are also due to Warren A. Hunt
Jr. and Matthew J. Kaufmann for their guidance.

\bibliographystyle{splncs}
\bibliography{references}

\end{document}

% titlepage-demo.tex
\documentclass{beamer}
\usepackage{graphicx}
\usepackage{tikz}
\usetikzlibrary{matrix}
% items enclosed in square brackets are optional; explanation below
\title{Verifying the FAT32 filesystem in ACL2}
\author{Mihir Mehta}
\institute{
  Department of Computer Science\\
  University of Texas at Austin\\[1ex]
  \texttt{mihir@cs.utexas.edu}
}
\date{06 April, 2018}

\AtBeginSection[]
{
  \begin{frame}<beamer>
    \frametitle{Outline}
    \tableofcontents[currentsection]
  \end{frame}
}

\addtobeamertemplate{navigation symbols}{}{%
    \usebeamerfont{footline}%
    \usebeamercolor[fg]{footline}%
    \hspace{1em}%
    \large \insertframenumber/\inserttotalframenumber
}

\begin{document}

%--- the titlepage frame -------------------------%
\begin{frame}[plain]
  \titlepage
\end{frame}

\begin{frame}{Outline}
  \tableofcontents
\end{frame}

%--- the presentation begins here ----------------%

\section{Motivation and related work}

\begin{frame}{Why we need a verified filesystem}
  \begin{itemize}
  \item Filesystems are everywhere, even as operating systems move
    towards making them invisible.
  \item In the absence of a clear specification of filesystems, users
    (and sysadmins in particular) are underserved.
  \item Modern filesystems have become increasingly complex, and so
    have the tools to analyse and recover data from them.
  \item It would be worthwhile to specify and formally verify, in the
    ACL2 theorem prover, the guarantees claimed by filesystems and tools.
  \end{itemize}
\end{frame}

\begin{frame}{Related work}
  \begin{itemize}
  \item In Haogang Chen's 2016 dissertation, the author uses Coq to
    build a filesystem (named FSCQ) which is proven safe against
    crashes in a new logical framework named Crash Hoare Logic. His
    (exported) Haskell implementation performs comparably to ext4.
  \item Hyperkernel (Nelson et al., SOSP '17) is a "push-button"
    verification effort, but approximates by changing POSIX system
    calls for ease of verification.
  \item In our work, we instead aim to model an existing filesystem (FAT32)
    faithfully and match the resulting disk image byte-to-byte.
  \end{itemize}
\end{frame}

\section{Our approach}

\end{document}

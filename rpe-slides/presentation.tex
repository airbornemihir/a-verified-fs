% titlepage-demo.tex
\documentclass{beamer}
\usepackage{graphicx}
% items enclosed in square brackets are optional; explanation below
\title{Verifying filesystems in ACL2}
\subtitle{Towards verifying file recovery tools}
\author{Mihir Mehta}
\institute{
  Department of Computer Science\\
  University of Texas at Austin\\[1ex]
  \texttt{mihir@cs.utexas.edu}
}
\date{09 March, 2017}

\begin{document}

%--- the titlepage frame -------------------------%
\begin{frame}[plain]
  \titlepage
\end{frame}

%--- the presentation begins here ----------------%
\begin{frame}{Overview}
  \begin{enumerate}
  \item Why we need a verified filesystem
  \item Our approach
  \item Progress so far
  \item Future work
  \end{enumerate}
\end{frame}

\begin{frame}{Why we need a verified filesystem}
  \begin{itemize}
  \item Filesystems are everywhere.
  \item Yet they're poorly understood - especially by people who
    should.
  \item Modern filesystems have become increasingly complex, and so
    have the tools to analyse and recover data from them.
  \item It might be nice, it might be nice to verify that the
    filesystems and the tools actually provide the guarantees they
    claim to provide.
  \end{itemize}
\end{frame}

\begin{frame}{What we need}
  \begin{itemize}
    \item Our filesystem should offer a set of operations that are
      sufficient for running a workload.
    \item However, as theorem proving researchers, we are loath to
      construct more operations than necessary - so what's the minimal
      set?
    \item We could attempt to emulate the VFS and replicate the
      operations for inodes, dentries, and files.
    \item That would
      mean 19 inode operations, 6 dentry operations and 22 file operations.
  \end{itemize}
\end{frame}

%% \begin{frame}[fragile]
%% \begin{verbatim}
%% struct inode_operations {
%%         int (*create) (struct inode *, struct dentry *, int);
%%         struct dentry * (*lookup) (struct inode *, struct dentry *);
%%         int (*link) (struct dentry *, struct inode *, struct dentry *);
%%         int (*unlink) (struct inode *, struct dentry *);
%%         int (*symlink) (struct inode *, struct dentry *, const char *);
%%         int (*mkdir) (struct inode *, struct dentry *, int);
%%         int (*rmdir) (struct inode *, struct dentry *);
%%         int (*mknod) (struct inode *, struct dentry *, int, dev_t);
%%         int (*rename) (struct inode *, struct dentry *, struct inode *, struct dentry *);
%%         int (*readlink) (struct dentry *, char *,int);
%%         int (*follow_link) (struct dentry *, struct nameidata *);
%%         void (*truncate) (struct inode *);
%%         int (*permission) (struct inode *, int);
%%         int (*setattr) (struct dentry *, struct iattr *);
%%         int (*getattr) (struct vfsmount *mnt, struct dentry *, struct kstat *);
%%         int (*setxattr) (struct dentry *, const char *, const void *, size_t, int);
%%         ssize_t (*getxattr) (struct dentry *, const char *, void *, size_t);
%%         ssize_t (*listxattr) (struct dentry *, char *, size_t);
%%         int (*removexattr) (struct dentry *, const char *);
%% };
%% \end{verbatim}
%% \end{frame}

\begin{frame}{Minimal set of operations?}
  \begin{itemize}
  \item There might be a better way, based on the Google file system.
  \item Here, we have a minimal set of operations:
    \begin{itemize}
    \item \texttt{create}
    \item \texttt{delete}
    \item \texttt{open}
    \item \texttt{close}
    \item \texttt{read}
    \item \texttt{write}
    \end{itemize}
  \item Further, we could leave \texttt{open} and \texttt{close} for
    the time when we want to deal with multiprogramming and
    concurrency.
  \item Thus, we have a minimal set of filesystem operations which we
    can model.
  \end{itemize}
\end{frame}

\begin{frame}{Modelling a filesystem}
  \begin{itemize}
  \item What should the filesystem look like?
  \item We're used to thinking of the filesystem as a tree... how
    about that?
  \item Thinking along the lines of recursive datatypes, an \texttt{alist}
    containing only strings or similar \texttt{alist}s in its
    \texttt{strip-cdrs} could do the job.
  \item The \texttt{strip-cars} would contain the file/directory
    names.
  \item Next, we'll look at a running example where we see what it
    looks like to add/delete files from such a model.
  \end{itemize}
\end{frame}

\begin{frame}{Model 1}
  \begin{figure}
    \centering
    \def\svgwidth{\columnwidth}
    \input{l1-1.pdf_tex}
  \end{figure}
\end{frame}

\begin{frame}{Model 1}
  \begin{figure}
    \centering
    \def\svgwidth{\columnwidth}
    \input{l1-2.pdf_tex}
  \end{figure}
\end{frame}

\begin{frame}{Model 1}
  \begin{figure}
    \centering
    \def\svgwidth{\columnwidth}
    \input{l1-3.pdf_tex}
  \end{figure}
\end{frame}

\begin{frame}{Model 1}
  \begin{figure}
    \centering
    \def\svgwidth{\columnwidth}
    \input{l1-4.pdf_tex}
  \end{figure}
\end{frame}

\begin{frame}{Model 2}
  \begin{itemize}
  \item Model 1 can hold unbounded text files and nested directory
    structures.
  \item However, there's no metadata, either to provide additional
    information or to validate the contents of the file.
  \item With an extra field for length, we can create a simple
    version of fsck that checks file contents for consistency, and
    verify that create, delete etc preserve this notion of
    consistency.
  \end{itemize}
\end{frame}

\begin{frame}{Model 2}
  \begin{figure}
    \centering
    \def\svgwidth{1.1\columnwidth}
    \input{l2-1.pdf_tex}
  \end{figure}
\end{frame}

\begin{frame}{Model 2}
  \begin{figure}
    \centering
    \def\svgwidth{1.1\columnwidth}
    \input{l2-2.pdf_tex}
  \end{figure}
\end{frame}

\begin{frame}{Model 2}
  \begin{figure}
    \centering
    \def\svgwidth{1.1\columnwidth}
    \input{l2-3.pdf_tex}
  \end{figure}
\end{frame}

\begin{frame}{Model 2}
  \begin{figure}
    \centering
    \def\svgwidth{1.1\columnwidth}
    \input{l2-4.pdf_tex}
  \end{figure}
\end{frame}

\begin{frame}{Model 3}
  \begin{itemize}
  \item As the next step, we would like to begin externalising the
    storage of file contents.
  \item It would also be good to break up file contents into "blocks"
    of a finite length.
    \begin{itemize}
    \item Note: this would mean storing file length is no longer
      optional.
    \end{itemize}
  \end{itemize}
\end{frame}

\begin{frame}{Model 3}
  \begin{figure}
    %% \centering
    \def\svgwidth{0.7\columnwidth}
    \input{l3-1.pdf_tex}
  \end{figure}
  \begin{table}[]
    %% \centering
    \caption{Disk}
    \label{my-label}
    \begin{tabular}{|l|}
      \hline
      \textbackslash0\textbackslash0\textbackslash0   \\ \hline
      Sun 19:0 \\ \hline
      0        \\ \hline
    \end{tabular}
  \end{table}
\end{frame}

\begin{frame}{Model 3}
  \begin{figure}
    %% \centering
    \def\svgwidth{0.7\columnwidth}
    \input{l3-2.pdf_tex}
  \end{figure}
  \begin{table}[]
    \centering
    \caption{Disk}
    \label{my-label}
    \begin{tabular}{|l|}
      \hline
      \textbackslash0\textbackslash0\textbackslash0   \\ \hline
      Sun 19:0 \\ \hline
      0        \\ \hline
      Tue 21:0 \\ \hline
      0        \\ \hline
    \end{tabular}
  \end{table}
\end{frame}

\begin{frame}{Model 3}
  \begin{figure}
    %% \centering
    \def\svgwidth{0.7\columnwidth}
    \input{l3-3.pdf_tex}
  \end{figure}
  \begin{table}[]
    \centering
    \caption{Disk}
    \label{my-label}
    \begin{tabular}{|l|}
      \hline
      \textbackslash0\textbackslash0\textbackslash0   \\ \hline
      Sun 19:0 \\ \hline
      0        \\ \hline
      Tue 21:0 \\ \hline
      0        \\ \hline
    \end{tabular}
  \end{table}
\end{frame}

\begin{frame}{Model 3}
  \begin{figure}
    %% \centering
    \def\svgwidth{0.7\columnwidth}
    \input{l3-4.pdf_tex}
  \end{figure}
  \begin{table}[]
    \centering
    \caption{Disk}
    \label{my-label}
    \begin{tabular}{|l|}
      \hline
      \textbackslash0\textbackslash0\textbackslash0   \\ \hline
      Sun 19:0 \\ \hline
      0        \\ \hline
      Tue 21:0 \\ \hline
      0        \\ \hline
      Wed 01:0 \\ \hline
      0        \\ \hline
    \end{tabular}
  \end{table}
\end{frame}

\begin{frame}{Proof approaches and techniques}
  \begin{itemize}
  \item In the fourth model, we implement garbage collection in the
    form of an allocation vector.
  \item What guarantees do we need to show that a filesystem of this
    kind is consistent?
  \end{itemize}
\end{frame}

\begin{frame}{Model 4}
  \begin{figure}
    %% \centering
    \def\svgwidth{0.7\columnwidth}
    \input{l4-1.pdf_tex}
  \end{figure}
  \begin{table}[]
    %% \centering
    \caption{Disk}
    \label{my-label}
    \begin{tabular}{|l|}
      \hline
      \textbackslash0\textbackslash0\textbackslash0   \\ \hline
      Sun 19:0 \\ \hline
      0        \\ \hline
    \end{tabular}
  \end{table}
\end{frame}

\begin{frame}{Model 4}
  \begin{figure}
    %% \centering
    \def\svgwidth{0.7\columnwidth}
    \input{l4-2.pdf_tex}
  \end{figure}
  \begin{table}[]
    \centering
    \caption{Disk}
    \label{my-label}
    \begin{tabular}{|l|}
      \hline
      \textbackslash0\textbackslash0\textbackslash0   \\ \hline
      Sun 19:0 \\ \hline
      0        \\ \hline
      Tue 21:0 \\ \hline
      0        \\ \hline
    \end{tabular}
  \end{table}
\end{frame}

\begin{frame}{Model 4}
  \begin{figure}
    %% \centering
    \def\svgwidth{0.7\columnwidth}
    \input{l4-3.pdf_tex}
  \end{figure}
  \begin{table}[]
    \centering
    \caption{Disk}
    \label{my-label}
    \begin{tabular}{|l|}
      \hline
      \textbackslash0\textbackslash0\textbackslash0   \\ \hline
      Sun 19:0 \\ \hline
      0        \\ \hline
      Tue 21:0 \\ \hline
      0        \\ \hline
    \end{tabular}
  \end{table}
\end{frame}

\begin{frame}{Model 4}
  \begin{figure}
    %% \centering
    \def\svgwidth{0.7\columnwidth}
    \input{l4-4.pdf_tex}
  \end{figure}
  \begin{table}[]
    \centering
    \caption{Disk}
    \label{my-label}
    \begin{tabular}{|l|}
      \hline
      \textbackslash0\textbackslash0\textbackslash0   \\ \hline
      Wed 01:0 \\ \hline
      0        \\ \hline
      Tue 21:0 \\ \hline
      0        \\ \hline
    \end{tabular}
  \end{table}
\end{frame}

\begin{frame}{Proof approaches and techniques}
  \begin{itemize}
  \item There are many properties that could be considered for
    correctness, but the read-over-write theorems from the first-order
    theory of arrays seem like a good place to start.
    \begin {enumerate}
    \item Reading from a location after writing to the same location
      should yield the data that was written.
    \item Reading from a location after writing to a different
      location should yield the same result as reading before writing.
    \end {enumerate}
    \item For each of the models 1, 2 and 3, we have proofs of correctness of
      the two read-after-write properties, based on the proofs of
      equivalence between each model and its successor.
  \end{itemize}
\end{frame}

\begin{frame}{Proof approaches and techniques}
  \begin{enumerate}
  \item For model 4, The disk and the allocation vector must be in harmony
    initially and updated in lockstep.
  \item Every block referred to in the filesystem must be marked
    "used" in the allocation vector.\\
    \textit{What about the complementary problem - making sure unused
      blocks are unmarked?}
  \item If n blocks are available in the allocation vector, the
    allocation algorithm must provide n blocks when requested.
  \item No matter how many blocks are returned by the allocation
    algorithm, they must be unique and disjoint with the blocks
    allocated to other files.
  \end{enumerate}
\end{frame}

\begin{frame}{Future work}
  \begin{itemize}
  \item Finish finitising the length of the disk and garbage
    collecting disk blocks that are left unused after a write or a delete
    operation.
  \item Possibly, add the system call open and close with the
    introduction of file descriptors.\\
    \textit{This would be a step towards the study of concurrent FS operations.}
  \item Linearise the tree, leaving only the disk.
  \item Eventually emulate the CP/M filesystem as a convincing proof
    of concept, and move on to fsck and file recovery tools.
  \end{itemize}
\end{frame}

\begin{frame}{Related work}
  \begin{itemize}
  \item In Haogang Chen's 2016 dissertation, the author uses Coq to
    build a filesystem (named FSCQ) which is proven safe against
    crashes.
  \item His implementation was exported into Haskell, and showed
    comparable performance to ext4 when run on FUSE.
  \item Our work is different - we're building verified models of
    actual filesystems with binary compatibility as the aim.
  \end{itemize}
\end{frame}

\end{document}
